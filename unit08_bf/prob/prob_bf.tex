\documentclass[11pt]{article}

\usepackage{fullpage}
\usepackage{amsmath, amssymb, bm, cite, epsfig, psfrag}
\usepackage{graphicx}
\usepackage{float}
\usepackage{amsthm}
\usepackage{amsfonts}
\usepackage{listings}
\usepackage{cite}
\usepackage{hyperref}
\usepackage{tikz}
\usepackage{enumitem}
\usetikzlibrary{shapes,arrows}
\usepackage{mdframed}
\usepackage{mcode}
\usepackage{siunitx}
\usepackage{mathtools}
\usepackage{physics}
%\usetikzlibrary{dsp,chains}

%\restylefloat{figure}
%\theoremstyle{plain}      \newtheorem{theorem}{Theorem}
%\theoremstyle{definition} \newtheorem{definition}{Definition}

\def\del{\partial}
\def\ds{\displaystyle}
\def\ts{\textstyle}
\def\beq{\begin{equation}}
\def\eeq{\end{equation}}
\def\beqa{\begin{eqnarray}}
\def\eeqa{\end{eqnarray}}
\def\beqan{\begin{eqnarray*}}
\def\eeqan{\end{eqnarray*}}
\def\nn{\nonumber}
\def\binomial{\mathop{\mathrm{binomial}}}
\def\half{{\ts\frac{1}{2}}}
\def\Half{{\frac{1}{2}}}
\def\N{{\mathbb{N}}}
\def\Z{{\mathbb{Z}}}
\def\Q{{\mathbb{Q}}}
\def\R{{\mathbb{R}}}
\def\C{{\mathbb{C}}}
\def\argmin{\mathop{\mathrm{arg\,min}}}
\def\argmax{\mathop{\mathrm{arg\,max}}}
%\def\span{\mathop{\mathrm{span}}}
\def\diag{\mathop{\mathrm{diag}}}
\def\x{\times}
\def\limn{\lim_{n \rightarrow \infty}}
\def\liminfn{\liminf_{n \rightarrow \infty}}
\def\limsupn{\limsup_{n \rightarrow \infty}}
\def\MID{\,|\,}
\def\MIDD{\,;\,}

\newtheorem{proposition}{Proposition}
\newtheorem{definition}{Definition}
\newtheorem{theorem}{Theorem}
\newtheorem{lemma}{Lemma}
\newtheorem{corollary}{Corollary}
\newtheorem{assumption}{Assumption}
\newtheorem{claim}{Claim}
\def\qed{\mbox{} \hfill $\Box$}
\setlength{\unitlength}{1mm}

\def\bhat{\widehat{b}}
\def\ehat{\widehat{e}}
\def\phat{\widehat{p}}
\def\qhat{\widehat{q}}
\def\rhat{\widehat{r}}
\def\shat{\widehat{s}}
\def\uhat{\widehat{u}}
\def\ubar{\overline{u}}
\def\vhat{\widehat{v}}
\def\xhat{\widehat{x}}
\def\xbar{\overline{x}}
\def\zhat{\widehat{z}}
\def\zbar{\overline{z}}
\def\la{\leftarrow}
\def\ra{\rightarrow}
\def\MSE{\mbox{\small \sffamily MSE}}
\def\SNR{\mbox{\small \sffamily SNR}}
\def\SINR{\mbox{\small \sffamily SINR}}
\def\arr{\rightarrow}
\def\Exp{\mathbb{E}}
\def\var{\mbox{var}}
\def\Tr{\mbox{Tr}}
\def\tm1{t\! - \! 1}
\def\tp1{t\! + \! 1}

\def\Xset{{\cal X}}

\newcommand{\one}{\mathbf{1}}
\newcommand{\abf}{\mathbf{a}}
\newcommand{\bbf}{\mathbf{b}}
\newcommand{\dbf}{\mathbf{d}}
\newcommand{\ebf}{\mathbf{e}}
\newcommand{\gbf}{\mathbf{g}}
\newcommand{\hbf}{\mathbf{h}}
\newcommand{\pbf}{\mathbf{p}}
\newcommand{\pbfhat}{\widehat{\mathbf{p}}}
\newcommand{\qbf}{\mathbf{q}}
\newcommand{\qbfhat}{\widehat{\mathbf{q}}}
\newcommand{\rbf}{\mathbf{r}}
\newcommand{\rbfhat}{\widehat{\mathbf{r}}}
\newcommand{\sbf}{\mathbf{s}}
\newcommand{\sbfhat}{\widehat{\mathbf{s}}}
\newcommand{\ubf}{\mathbf{u}}
\newcommand{\ubfhat}{\widehat{\mathbf{u}}}
\newcommand{\utildebf}{\tilde{\mathbf{u}}}
\newcommand{\vbf}{\mathbf{v}}
\newcommand{\vbfhat}{\widehat{\mathbf{v}}}
\newcommand{\wbf}{\mathbf{w}}
\newcommand{\wbfhat}{\widehat{\mathbf{w}}}
\newcommand{\xbf}{\mathbf{x}}
\newcommand{\xbfhat}{\widehat{\mathbf{x}}}
\newcommand{\xbfbar}{\overline{\mathbf{x}}}
\newcommand{\ybf}{\mathbf{y}}
\newcommand{\zbf}{\mathbf{z}}
\newcommand{\zbfbar}{\overline{\mathbf{z}}}
\newcommand{\zbfhat}{\widehat{\mathbf{z}}}
\newcommand{\Ahat}{\widehat{A}}
\newcommand{\Abf}{\mathbf{A}}
\newcommand{\Bbf}{\mathbf{B}}
\newcommand{\Cbf}{\mathbf{C}}
\newcommand{\Bbfhat}{\widehat{\mathbf{B}}}
\newcommand{\Dbf}{\mathbf{D}}
\newcommand{\Ebf}{\mathbf{E}}
\newcommand{\Gbf}{\mathbf{G}}
\newcommand{\Hbf}{\mathbf{H}}
\newcommand{\Ibf}{\mathbf{I}}
\newcommand{\Kbf}{\mathbf{K}}
\newcommand{\Pbf}{\mathbf{P}}
\newcommand{\Phat}{\widehat{P}}
\newcommand{\Qbf}{\mathbf{Q}}
\newcommand{\Rbf}{\mathbf{R}}
\newcommand{\Rhat}{\widehat{R}}
\newcommand{\Sbf}{\mathbf{S}}
\newcommand{\Ubf}{\mathbf{U}}
\newcommand{\Vbf}{\mathbf{V}}
\newcommand{\Wbf}{\mathbf{W}}
\newcommand{\Xhat}{\widehat{X}}
\newcommand{\Xbf}{\mathbf{X}}
\newcommand{\Ybf}{\mathbf{Y}}
\newcommand{\Zbf}{\mathbf{Z}}
\newcommand{\Zhat}{\widehat{Z}}
\newcommand{\Zbfhat}{\widehat{\mathbf{Z}}}
\def\alphabf{{\boldsymbol \alpha}}
\def\betabf{{\boldsymbol \beta}}
\def\mubf{{\boldsymbol \mu}}
\def\lambdabf{{\boldsymbol \lambda}}
\def\etabf{{\boldsymbol \eta}}
\def\xibf{{\boldsymbol \xi}}
\def\taubf{{\boldsymbol \tau}}
\def\sigmahat{{\widehat{\sigma}}}
\def\thetabf{{\bm{\theta}}}
\def\thetabfhat{{\widehat{\bm{\theta}}}}
\def\thetahat{{\widehat{\theta}}}
\def\mubar{\overline{\mu}}
\def\muavg{\mu}
\def\sigbf{\bm{\sigma}}
\def\etal{\emph{et al.}}
\def\Ggothic{\mathfrak{G}}
\def\Pset{{\mathcal P}}
\newcommand{\bigCond}[2]{\bigl({#1} \!\bigm\vert\! {#2} \bigr)}
\newcommand{\BigCond}[2]{\Bigl({#1} \!\Bigm\vert\! {#2} \Bigr)}

\def\Rect{\mathop{Rect}}
\def\sinc{\mathop{sinc}}
\def\NF{\mathrm{NF}}
\def\Real{\mathrm{Re}}
\def\Imag{\mathrm{Im}}
\newcommand{\tran}{^{\text{\sf T}}}
\newcommand{\herm}{^{\text{\sf H}}}


% Solution environment
\definecolor{lightgray}{gray}{0.95}
\newmdenv[linecolor=white,backgroundcolor=lightgray,frametitle=Solution:]{solution}



\begin{document}

\title{Problems:  Antenna Arrays and Beamforming\\
ECE-GY 6023. Wireless Communications}
\author{Prof.\ Sundeep Rangan}
\date{}

\maketitle


\begin{enumerate}

\item \emph{ULA.}  Consider a ULA with spatial signature,
\[
    \ubf(\phi) = \left[ 1, e^{i \beta \sin(\phi)}, \cdots,
    e^{i \beta(N-1) \sin(\phi)} \right]\tran, \quad
    \beta = \frac{2\pi d}{\lambda},
\]
where $\phi$ is the azimuth angle relative to boresight.
\begin{enumerate}[label=(\alph*)]
\item Find the optimal beamforming vector for an angle $\phi= 30^\circ$
with $N=8$ antennas and $d=\lambda/2$.
\item Now suppose a path arrives at an angle $\phi = 60^\circ$.
What is the beamforming gain on that path using the BF vector in part (a)?
\end{enumerate}

\item \emph{Constrained beamforming.} Consider a SIMO channel,
\[
    \rbf = \hbf x + \vbf, \quad \vbf \sim C{\mathcal N}(0,N_0 \Ibf), \quad |x|^2 = E_x.
\]
\begin{enumerate}[label=(\alph*)]
\item What is the SNR after beamforming with a vector $\wbf$?
\item What is the maximum SNR after beamforming if we are allowed any vector $\wbf$?
\item What is the maximum SNR after beamforming if the components of $\wbf$
must be constant magnitude, $|w_n| = 1$.  That is, you can only change the phase of $w_n$.
This commonly occurs in analog beamforming systems.
\item What is the maximum SNR after beamforming if
$w_n \neq 0$ for only one antenna $n$.  This is called antenna selection beamforming.
\item Suppose $\hbf = [4,2+i,-1,i]\tran$ and $E_x/N_0=$\, \si{5}{dB}.  What is the SNR
after beamforming in parts (b), (c) and (d)?
\end{enumerate}


\item \label{prob:freq_resp}
\emph{Vector frequency response}.  Consider a SIMO channel where
\[
    \rbf = \hbf(f)\xbf + \vbf, \quad \vbf \sim C{\mathcal N}(0,N_0 \Ibf), \quad |x|^2 = E_x,
\]
where $\hbf(f)$ is the frequency-dependent channel
\[
    \hbf = \sum_{\ell=1}^L a_\ell e^{2\pi if \tau_\ell} \ubf(\theta_\ell),
\]
and $L$ is the number of paths. Suppose the
array is $N=8$ ULA with antenna spacing $d=\lambda/2$ and the channel
has three paths with parameters shown in Table \ref{tab:freq_resp}.
In the table, the column ``SNR" is $E_x|a_\ell|^2/N_0$, and the column ``path phase" is the angle of $a_\ell$.

\begin{table}
\centering
\footnotesize
\begin{tabular}{|l|l|l|l|l|}
\hline
Path & SNR [dB] & Path phase [deg] & AoA $\phi_\ell$ [deg] & Delay $\tau_\ell$ [ns]  \\ \hline
1 & 4 & 0 & 30 & 0 \\ \hline
2 & 1 & 180 & -30 & 100 \\ \hline
3 & -2 & 65 & 80 & 130 \\ \hline
\end{tabular}
\caption{Problem \ref{prob:freq_resp}:  Path parameters.}
\label{tab:freq_resp}
\end{table}

Write a short MATLAB program to plot the following SNR values for 100
frequency points in the range $f \in [-10,10]$\, \si{MHz}:
\begin{enumerate}[label=(\alph*)]
\item The SNR only using the signal from antenna 1.
\item The SNR after beamforming where the beamforming vector is optimally selected at each
frequency.
\item The SNR after beamforming where the beamforming vector is optimized for $f=0$.
\end{enumerate}

\item \emph{Beamforming with SVDs}.  Below are a set of channel matrices $\Hbf$
described by TX and RX spatial signatures $\ubf_{\rm tx}(\Omega^{\rm tx}_\ell)$
and $\ubf_{\rm rx}(\Omega^{\rm rx}_\ell)$ and complex gains
$g_\ell$.
For each of the  channel matrices,
find the rank $r$, the maximum singular value and the optimal TX and RX
beamforming vectors.  Assume all matrices have dimensions
 $\Hbf \in \C^{N_r \times N_t}$,
and the spatial signatures are normalized so that
\[
    \|\ubf_{\rm tx}(\Omega^{\rm tx})\|^2=N_t, \quad
    \|\ubf_{\rm rx}(\Omega^{\rm rx})\|^2=N_r,
\]
for all angles $\Omega^{\rm tx}$ and $\Omega^{\rm rx}$.
\begin{enumerate}[label=(\alph*)]
\item Single path channel:
\[
    \Hbf = g_1
        \ubf_{\rm rx}(\Omega_1^{\rm rx}) \ubf_{\rm tx}(\Omega_1^{\rm tx}).
\]
\item Two channel paths with same RX angle $\Omega_1^{\rm rx}$
and two different TX angles, $\Omega_1^{\rm tx}$ and $\Omega_2^{\rm tx}$:
\[
    \Hbf = g_1
        \ubf_{\rm rx}(\Omega_1^{\rm rx}) \ubf_{\rm tx}(\Omega_1^{\rm tx}) +
        g_2 \ubf_{\rm rx}(\Omega_1^{\rm rx}) \ubf_{\rm tx}(\Omega_2^{\rm tx}).
\]
Assume $|g_1| > |g_2|$ and $\ubf_{\rm tx}(\Omega_1^{\rm tx}) \perp \ubf_{\rm tx}(\Omega_2^{\rm tx})$

\item Two channel paths with two different RX angles
and two different TX angles:
\[
    \Hbf = g_1
        \ubf_{\rm rx}(\Omega_1^{\rm rx}) \ubf_{\rm tx}(\Omega_1^{\rm tx}) +
        g_2 \ubf_{\rm rx}(\Omega_2^{\rm rx}) \ubf_{\rm tx}(\Omega_2^{\rm tx}).
\]
Assume $|g_1| > |g_2|$, $\ubf_{\rm rx}(\Omega_1^{\rm rx}) \perp \ubf_{\rm rx}(\Omega_2^{\rm rx})$ and $\ubf_{\rm tx}(\Omega_1^{\rm tx}) \perp \ubf_{\rm tx}(\Omega_2^{\rm tx})$.

\item Same as part (b) except the two TX spatial signatures are not orthongonal:
\[
    \ubf_{\rm tx}(\Omega_1^{\rm tx})^*\ubf_{\rm tx}(\Omega_2^{\rm tx})  =\rho N_t,
\]
for some $\rho$ with $|\rho|\leq 1$.

\end{enumerate}

\item \emph{Array normalization.}  When there is no mutual coupling,
an array with $N$ elements can generally obtain a gain of $N$ (in linear scale).
However, in this exercise, we will show that the gain is more limited when the
antennas are closely spaced.  To this end, consider a  ULA with $N$ elements
with total length $L$, so the antennas are spaced by $L/N$.
Given a beamforming vector $\wbf \in \C^N$,
suppose the power intensity at angle $\phi$ is,
\[
    U(\phi) = c|\wbf\tran \ubf(\phi)|^2,
\]
where $\ubf(\phi)$ is the spatial signature and $c > 0$ is some constant.
Suppose we use the all ones beamforming vector $w_n = \frac{1}{N}$ so that the
energy is maximized at $\phi = 0$.
\begin{enumerate}[label=(\alph*)]
\item Write the spatial signature $\ubf(\phi)$ as a function of the total array length
$L$, wavelength $\lambda$ and number of elements $N$.

\item Find the limit
\[
    \lim_{N \rightarrow \infty} U(\phi).
\]

\item Recall the antenna directivity is
\[
    D(\phi) = \frac{4\pi U(\phi)}{P_{\rm rad}},
\]
where $P_{\rm rad}$ is the total radiated power,
\[
    P_{\rm rad} = \int_{-\pi/2}^{\pi/2} \int_{-\pi}^{\pi} U(\phi)\cos \theta \,
        \dd \phi \,\dd \theta.
\]
Use MATLAB to plot $D(\phi)$ using the limit of large numbers of antennas for $L=2\lambda$,
$4\lambda$ and $8\lambda$.  What is the peak gain in each case.


\end{enumerate}




\end{enumerate}





\end{document}

slide 8 14 def p0 p1

slide 12 index i
multiple class
bring back bayes risk

25 y-yhat
30 add numbers

create one hot
another ex for one SE rule
remove 
param est
review PCA

add error slide 9

covert risk min side to slide.  C and L

compress ridge / lasso to one slide
beta ridge formula

Use SVD







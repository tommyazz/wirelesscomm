\documentclass[11pt]{article}

\usepackage{fullpage}
\usepackage{amsmath, amssymb, bm, cite, epsfig, psfrag}
\usepackage{graphicx}
\usepackage{float}
\usepackage{amsthm}
\usepackage{amsfonts}
\usepackage{listings}
\usepackage{cite}
\usepackage{hyperref}
\usepackage{tikz}
\usepackage{enumitem}
\usetikzlibrary{shapes,arrows}
\usepackage{mdframed}
\usepackage{mcode}
\usepackage{siunitx}
%\usetikzlibrary{dsp,chains}

%\restylefloat{figure}
%\theoremstyle{plain}      \newtheorem{theorem}{Theorem}
%\theoremstyle{definition} \newtheorem{definition}{Definition}

\def\del{\partial}
\def\ds{\displaystyle}
\def\ts{\textstyle}
\def\beq{\begin{equation}}
\def\eeq{\end{equation}}
\def\beqa{\begin{eqnarray}}
\def\eeqa{\end{eqnarray}}
\def\beqan{\begin{eqnarray*}}
\def\eeqan{\end{eqnarray*}}
\def\nn{\nonumber}
\def\binomial{\mathop{\mathrm{binomial}}}
\def\half{{\ts\frac{1}{2}}}
\def\Half{{\frac{1}{2}}}
\def\N{{\mathbb{N}}}
\def\Z{{\mathbb{Z}}}
\def\Q{{\mathbb{Q}}}
\def\R{{\mathbb{R}}}
\def\C{{\mathbb{C}}}
\def\argmin{\mathop{\mathrm{arg\,min}}}
\def\argmax{\mathop{\mathrm{arg\,max}}}
%\def\span{\mathop{\mathrm{span}}}
\def\diag{\mathop{\mathrm{diag}}}
\def\x{\times}
\def\limn{\lim_{n \rightarrow \infty}}
\def\liminfn{\liminf_{n \rightarrow \infty}}
\def\limsupn{\limsup_{n \rightarrow \infty}}
\def\MID{\,|\,}
\def\MIDD{\,;\,}

\newtheorem{proposition}{Proposition}
\newtheorem{definition}{Definition}
\newtheorem{theorem}{Theorem}
\newtheorem{lemma}{Lemma}
\newtheorem{corollary}{Corollary}
\newtheorem{assumption}{Assumption}
\newtheorem{claim}{Claim}
\def\qed{\mbox{} \hfill $\Box$}
\setlength{\unitlength}{1mm}

\def\bhat{\widehat{b}}
\def\ehat{\widehat{e}}
\def\phat{\widehat{p}}
\def\qhat{\widehat{q}}
\def\rhat{\widehat{r}}
\def\shat{\widehat{s}}
\def\uhat{\widehat{u}}
\def\ubar{\overline{u}}
\def\vhat{\widehat{v}}
\def\xhat{\widehat{x}}
\def\xbar{\overline{x}}
\def\zhat{\widehat{z}}
\def\zbar{\overline{z}}
\def\la{\leftarrow}
\def\ra{\rightarrow}
\def\MSE{\mbox{\small \sffamily MSE}}
\def\SNR{\mbox{\small \sffamily SNR}}
\def\SINR{\mbox{\small \sffamily SINR}}
\def\arr{\rightarrow}
\def\Exp{\mathbb{E}}
\def\var{\mbox{var}}
\def\Tr{\mbox{Tr}}
\def\tm1{t\! - \! 1}
\def\tp1{t\! + \! 1}

\def\Xset{{\cal X}}

\newcommand{\one}{\mathbf{1}}
\newcommand{\abf}{\mathbf{a}}
\newcommand{\bbf}{\mathbf{b}}
\newcommand{\dbf}{\mathbf{d}}
\newcommand{\ebf}{\mathbf{e}}
\newcommand{\gbf}{\mathbf{g}}
\newcommand{\hbf}{\mathbf{h}}
\newcommand{\pbf}{\mathbf{p}}
\newcommand{\pbfhat}{\widehat{\mathbf{p}}}
\newcommand{\qbf}{\mathbf{q}}
\newcommand{\qbfhat}{\widehat{\mathbf{q}}}
\newcommand{\rbf}{\mathbf{r}}
\newcommand{\rbfhat}{\widehat{\mathbf{r}}}
\newcommand{\sbf}{\mathbf{s}}
\newcommand{\sbfhat}{\widehat{\mathbf{s}}}
\newcommand{\ubf}{\mathbf{u}}
\newcommand{\ubfhat}{\widehat{\mathbf{u}}}
\newcommand{\utildebf}{\tilde{\mathbf{u}}}
\newcommand{\vbf}{\mathbf{v}}
\newcommand{\vbfhat}{\widehat{\mathbf{v}}}
\newcommand{\wbf}{\mathbf{w}}
\newcommand{\wbfhat}{\widehat{\mathbf{w}}}
\newcommand{\xbf}{\mathbf{x}}
\newcommand{\xbfhat}{\widehat{\mathbf{x}}}
\newcommand{\xbfbar}{\overline{\mathbf{x}}}
\newcommand{\ybf}{\mathbf{y}}
\newcommand{\zbf}{\mathbf{z}}
\newcommand{\zbfbar}{\overline{\mathbf{z}}}
\newcommand{\zbfhat}{\widehat{\mathbf{z}}}
\newcommand{\Ahat}{\widehat{A}}
\newcommand{\Abf}{\mathbf{A}}
\newcommand{\Bbf}{\mathbf{B}}
\newcommand{\Cbf}{\mathbf{C}}
\newcommand{\Bbfhat}{\widehat{\mathbf{B}}}
\newcommand{\Dbf}{\mathbf{D}}
\newcommand{\Ebf}{\mathbf{E}}
\newcommand{\Gbf}{\mathbf{G}}
\newcommand{\Hbf}{\mathbf{H}}
\newcommand{\Kbf}{\mathbf{K}}
\newcommand{\Pbf}{\mathbf{P}}
\newcommand{\Phat}{\widehat{P}}
\newcommand{\Qbf}{\mathbf{Q}}
\newcommand{\Rbf}{\mathbf{R}}
\newcommand{\Rhat}{\widehat{R}}
\newcommand{\Sbf}{\mathbf{S}}
\newcommand{\Ubf}{\mathbf{U}}
\newcommand{\Vbf}{\mathbf{V}}
\newcommand{\Wbf}{\mathbf{W}}
\newcommand{\Xhat}{\widehat{X}}
\newcommand{\Xbf}{\mathbf{X}}
\newcommand{\Ybf}{\mathbf{Y}}
\newcommand{\Zbf}{\mathbf{Z}}
\newcommand{\Zhat}{\widehat{Z}}
\newcommand{\Zbfhat}{\widehat{\mathbf{Z}}}
\def\alphabf{{\boldsymbol \alpha}}
\def\betabf{{\boldsymbol \beta}}
\def\mubf{{\boldsymbol \mu}}
\def\lambdabf{{\boldsymbol \lambda}}
\def\etabf{{\boldsymbol \eta}}
\def\xibf{{\boldsymbol \xi}}
\def\taubf{{\boldsymbol \tau}}
\def\sigmahat{{\widehat{\sigma}}}
\def\thetabf{{\bm{\theta}}}
\def\thetabfhat{{\widehat{\bm{\theta}}}}
\def\thetahat{{\widehat{\theta}}}
\def\mubar{\overline{\mu}}
\def\muavg{\mu}
\def\sigbf{\bm{\sigma}}
\def\etal{\emph{et al.}}
\def\Ggothic{\mathfrak{G}}
\def\Pset{{\mathcal P}}
\newcommand{\bigCond}[2]{\bigl({#1} \!\bigm\vert\! {#2} \bigr)}
\newcommand{\BigCond}[2]{\Bigl({#1} \!\Bigm\vert\! {#2} \Bigr)}

\def\Rect{\mathop{Rect}}
\def\sinc{\mathop{sinc}}
\def\NF{\mathrm{NF}}
\def\Real{\mathrm{Re}}
\def\Imag{\mathrm{Im}}
\newcommand{\tran}{^{\text{\sf T}}}
\newcommand{\herm}{^{\text{\sf H}}}


% Solution environment
\definecolor{lightgray}{gray}{0.95}
\newmdenv[linecolor=white,backgroundcolor=lightgray,frametitle=Solution:]{solution}



\begin{document}

\title{Problem Solutions:  Non-LOS Propagation and Link Budget Analysis\\
EL-GY 6023. Wireless Communications}
\author{Prof.\ Sundeep Rangan}
\date{}

\maketitle

In all the problems below, unless specified otherwise, $\phi$ is the
azimuth angle and $\theta$ is elevation angle.

\begin{enumerate}

\item \emph{Noise:}
Suppose a receiver consists of an low noise amplifier with a gain of 20 dB
and noise figure of 2 dB, followed by a second stage of amplification of
another 15 dB with a noise figure of 10 dB.
\begin{enumerate}[label=(\alph*)]
\item What is the total noise figure and gain of the system?
\item Suppose a 10 dB attenuator is placed at the input of the LNA.
What is the resulting overall gain and noise figure?
\item What if the attenuator is placed at the output of the LNA?
\end{enumerate}


\item \emph{SINR:}
Suppose when a transmitter, TX1, sends data to a receiver RX
without interference, the SNR is 10 dB.  Now suppose that TX2
starts transmitting resulting in interference.
Suppose TX2 transmits at the same power as TX1 and the path loss from
TX2 to RX is 5 dB greater than the path loss from TX1 to RX.
What is the resulting SINR from TX1 to RX when TX2 is transmitting?

\item \emph{Reflection loss:}  Consider a reflection at an 
interface going from a characteristic impedance $\eta_1$ to $\eta_2$ at 
an incident angle of $\theta_i = 0$.
\begin{enumerate}[label=(\alph*)]
\item What is the reflected angle $\theta_r$ and refracted 
angle $\theta_t$?
\item Show that the reflection coefficient $\Gamma$ is 
identical in the parallel and perpendicular polarizations.
Find $\Gamma$ in terms of $\eta_1$ and $\eta_2$.
\item Estimate the expected reflected power gain $|\Gamma|^2$ 
when a wave in free space strikes dry concrete.  
Assume concrete has a relative permittivity  of $\epsilon_r \approx 4.5$.
\end{enumerate}



\item \emph{SNR requirements:}
A signal is received at power of $P_{rx} = $-100 dBm and the noise power density
(including the noise figure) is $N_0 = $ -170 dBm/Hz.
If the receiver requires $E_b/N_0=$ \SI{6}{dB}, 
what is the minimum time to transmit $b=1000$ bits?


\item \emph{Simulating a statistical model:}
Write short MATLAB code to do the following.  You do not need to run the code,
just write the code.
\begin{enumerate}[label=(\alph*)]
\item Suppose \mcode{nrx=1000} RX locations are randomly located uniformly
in a circle of radius \mcode{rmax = 100}\, \si{m} from the origin.  
Generate a random vector \mcode{dist2} representing the  random distances
from the origin of the RX locations.
\item Assuming the transmitter is at the origin at a height \mcode{htx=2}\, \si{m} higher than the RX, compute the distances \mcode{dist} to the RXs.
\item Assuming a path loss model,
\[
    PL = 32.4 + 14.3\log_{10}(d \mbox{\, [m]}) + 20\log_{10}(f_c \mbox{\, [GHz]})  + \xi,
    \quad \xi \sim \mathcal{N}(0,\sigma^2), \mbox{ [dB]}
\]
generate random path losses to the RXs.   Assume $\sigma = 4$\, \si{dB} and
$f_c = 2.3$\, \si{GHz}.
\item Finally plot a CDF of $E_s/N_0$ with transmit power, \mcode{Ptx = 15}\,
\si{dBm}, bandwidth \mcode{B = } \SI{20}{MHz} and thermal noise \mcode{N0=-170}\, \si{dBm/Hz}.
\end{enumerate}



\item \emph{Simulating a statistical model:}
Consider an indoor propagation model where the path loss is,
\[
    PL = \mathrm{FSPL}(d) + L_0 N, \quad N \sim \mathrm{Poisson}(\alpha d),
\]
where $d$ is the distance between the TX and RX,
$\mathrm{FSPL}(d)$ is the free-space path loss, $N$ is
the number of walls between a transmitter and receiver and $N$ is modeled as
a Poisson random variable where $\alpha$ is average number of walls
per meter of distance.
Write a MATLAB function
\begin{lstlisting}
    function pl = indoorPL(d, ... )
\end{lstlisting}
to generate random path loss values as function of a vector of distances.
State all the other parameters this function needs.  You may assume you have
access to a function \mcode{fspl(lambda,d)} for the free-space path loss.
Place comments in your code describing all the input arguments.

\item \emph{Outage probability:} Suppose that a link has the following properties:
\begin{itemize}
\item TX power, $P_{tx} =$ 20 dBm
\item Bandwidth, $B =$ 20 MHz
\item Noise power density (including noise figure) $N_0=$ -170 dBm/Hz.
\end{itemize}
Answer the following:
\begin{enumerate}[label=(\alph*)]
\item What is the maximum path loss, $PL_{\rm max}$, that the link can
have to meet an SNR target of 10 dB?
\item Suppose that the path loss is lognormally distributed with
\[
    PL = PL_0 + \xi, \quad \xi \sim {\mathcal N}(0,\sigma^2),
\]
where $PL_0 =$ \SI{100}{dB} and $\sigma =$ \SI{8}{dB}.
What is the outage probability $P_{out}=\Pr(PL \geq PL_{\rm max})$
using the value $PL_{max}$ from part (a)?
Your answer should have
a $Q$-function.  You can evaluate it with MATLAB's function \mcode{qfunc}.

\item A common model for indoor path loss is given by
\[
    PL = PL_0 + \xi + DN, \quad \xi \sim {\mathcal N}(0,\sigma^2),
\]
where $N$ is the number of walls that the signal must pass through
and $D$ is the loss per wall.  Suppose that we model the number of
walls as a random variable with distribution:
\[
    P(N=n) = \begin{cases}
        0.5 & \mbox{if } n=0 \\
        0.3 & \mbox{if } n=1 \\
        0.2 & \mbox{if } n=2 \\
        0 & \mbox{else, }
        \end{cases}
\]
and the loss per wall is $D=$ \SI{7}{dB}.
What is the outage probability $P_{out}$?
\end{enumerate}


\end{enumerate}

\end{document}




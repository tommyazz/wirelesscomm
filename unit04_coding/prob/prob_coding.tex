\documentclass[11pt]{article}

\usepackage{fullpage}
\usepackage{amsmath, amssymb, bm, cite, epsfig, psfrag}
\usepackage{graphicx}
\usepackage{float}
\usepackage{amsthm}
\usepackage{amsfonts}
\usepackage{listings}
\usepackage{cite}
\usepackage{hyperref}
\usepackage{tikz}
\usepackage{enumitem}
\usetikzlibrary{shapes,arrows}
\usepackage{mdframed}
\usepackage{mcode}
\usepackage{siunitx}
%\usetikzlibrary{dsp,chains}

%\restylefloat{figure}
%\theoremstyle{plain}      \newtheorem{theorem}{Theorem}
%\theoremstyle{definition} \newtheorem{definition}{Definition}

\def\del{\partial}
\def\ds{\displaystyle}
\def\ts{\textstyle}
\def\beq{\begin{equation}}
\def\eeq{\end{equation}}
\def\beqa{\begin{eqnarray}}
\def\eeqa{\end{eqnarray}}
\def\beqan{\begin{eqnarray*}}
\def\eeqan{\end{eqnarray*}}
\def\nn{\nonumber}
\def\binomial{\mathop{\mathrm{binomial}}}
\def\half{{\ts\frac{1}{2}}}
\def\Half{{\frac{1}{2}}}
\def\N{{\mathbb{N}}}
\def\Z{{\mathbb{Z}}}
\def\Q{{\mathbb{Q}}}
\def\R{{\mathbb{R}}}
\def\C{{\mathbb{C}}}
\def\argmin{\mathop{\mathrm{arg\,min}}}
\def\argmax{\mathop{\mathrm{arg\,max}}}
%\def\span{\mathop{\mathrm{span}}}
\def\diag{\mathop{\mathrm{diag}}}
\def\x{\times}
\def\limn{\lim_{n \rightarrow \infty}}
\def\liminfn{\liminf_{n \rightarrow \infty}}
\def\limsupn{\limsup_{n \rightarrow \infty}}
\def\MID{\,|\,}
\def\MIDD{\,;\,}

\newtheorem{proposition}{Proposition}
\newtheorem{definition}{Definition}
\newtheorem{theorem}{Theorem}
\newtheorem{lemma}{Lemma}
\newtheorem{corollary}{Corollary}
\newtheorem{assumption}{Assumption}
\newtheorem{claim}{Claim}
\def\qed{\mbox{} \hfill $\Box$}
\setlength{\unitlength}{1mm}

\def\bhat{\widehat{b}}
\def\ehat{\widehat{e}}
\def\phat{\widehat{p}}
\def\qhat{\widehat{q}}
\def\rhat{\widehat{r}}
\def\shat{\widehat{s}}
\def\uhat{\widehat{u}}
\def\ubar{\overline{u}}
\def\vhat{\widehat{v}}
\def\xhat{\widehat{x}}
\def\xbar{\overline{x}}
\def\zhat{\widehat{z}}
\def\zbar{\overline{z}}
\def\la{\leftarrow}
\def\ra{\rightarrow}
\def\MSE{\mbox{\small \sffamily MSE}}
\def\SNR{\mbox{\small \sffamily SNR}}
\def\SINR{\mbox{\small \sffamily SINR}}
\def\arr{\rightarrow}
\def\Exp{\mathbb{E}}
\def\var{\mbox{var}}
\def\Tr{\mbox{Tr}}
\def\tm1{t\! - \! 1}
\def\tp1{t\! + \! 1}

\def\Xset{{\cal X}}

\newcommand{\one}{\mathbf{1}}
\newcommand{\abf}{\mathbf{a}}
\newcommand{\bbf}{\mathbf{b}}
\newcommand{\dbf}{\mathbf{d}}
\newcommand{\ebf}{\mathbf{e}}
\newcommand{\gbf}{\mathbf{g}}
\newcommand{\hbf}{\mathbf{h}}
\newcommand{\pbf}{\mathbf{p}}
\newcommand{\pbfhat}{\widehat{\mathbf{p}}}
\newcommand{\qbf}{\mathbf{q}}
\newcommand{\qbfhat}{\widehat{\mathbf{q}}}
\newcommand{\rbf}{\mathbf{r}}
\newcommand{\rbfhat}{\widehat{\mathbf{r}}}
\newcommand{\sbf}{\mathbf{s}}
\newcommand{\sbfhat}{\widehat{\mathbf{s}}}
\newcommand{\ubf}{\mathbf{u}}
\newcommand{\ubfhat}{\widehat{\mathbf{u}}}
\newcommand{\utildebf}{\tilde{\mathbf{u}}}
\newcommand{\vbf}{\mathbf{v}}
\newcommand{\vbfhat}{\widehat{\mathbf{v}}}
\newcommand{\wbf}{\mathbf{w}}
\newcommand{\wbfhat}{\widehat{\mathbf{w}}}
\newcommand{\xbf}{\mathbf{x}}
\newcommand{\xbfhat}{\widehat{\mathbf{x}}}
\newcommand{\xbfbar}{\overline{\mathbf{x}}}
\newcommand{\ybf}{\mathbf{y}}
\newcommand{\zbf}{\mathbf{z}}
\newcommand{\zbfbar}{\overline{\mathbf{z}}}
\newcommand{\zbfhat}{\widehat{\mathbf{z}}}
\newcommand{\Ahat}{\widehat{A}}
\newcommand{\Abf}{\mathbf{A}}
\newcommand{\Bbf}{\mathbf{B}}
\newcommand{\Cbf}{\mathbf{C}}
\newcommand{\Bbfhat}{\widehat{\mathbf{B}}}
\newcommand{\Dbf}{\mathbf{D}}
\newcommand{\Ebf}{\mathbf{E}}
\newcommand{\Gbf}{\mathbf{G}}
\newcommand{\Hbf}{\mathbf{H}}
\newcommand{\Kbf}{\mathbf{K}}
\newcommand{\Pbf}{\mathbf{P}}
\newcommand{\Phat}{\widehat{P}}
\newcommand{\Qbf}{\mathbf{Q}}
\newcommand{\Rbf}{\mathbf{R}}
\newcommand{\Rhat}{\widehat{R}}
\newcommand{\Sbf}{\mathbf{S}}
\newcommand{\Ubf}{\mathbf{U}}
\newcommand{\Vbf}{\mathbf{V}}
\newcommand{\Wbf}{\mathbf{W}}
\newcommand{\Xhat}{\widehat{X}}
\newcommand{\Xbf}{\mathbf{X}}
\newcommand{\Ybf}{\mathbf{Y}}
\newcommand{\Zbf}{\mathbf{Z}}
\newcommand{\Zhat}{\widehat{Z}}
\newcommand{\Zbfhat}{\widehat{\mathbf{Z}}}
\def\alphabf{{\boldsymbol \alpha}}
\def\betabf{{\boldsymbol \beta}}
\def\mubf{{\boldsymbol \mu}}
\def\lambdabf{{\boldsymbol \lambda}}
\def\etabf{{\boldsymbol \eta}}
\def\xibf{{\boldsymbol \xi}}
\def\taubf{{\boldsymbol \tau}}
\def\sigmahat{{\widehat{\sigma}}}
\def\thetabf{{\bm{\theta}}}
\def\thetabfhat{{\widehat{\bm{\theta}}}}
\def\thetahat{{\widehat{\theta}}}
\def\mubar{\overline{\mu}}
\def\muavg{\mu}
\def\sigbf{\bm{\sigma}}
\def\etal{\emph{et al.}}
\def\Ggothic{\mathfrak{G}}
\def\Pset{{\mathcal P}}
\newcommand{\bigCond}[2]{\bigl({#1} \!\bigm\vert\! {#2} \bigr)}
\newcommand{\BigCond}[2]{\Bigl({#1} \!\Bigm\vert\! {#2} \Bigr)}

\def\Rect{\mathop{Rect}}
\def\sinc{\mathop{sinc}}
\def\NF{\mathrm{NF}}
\def\Real{\mathrm{Re}}
\def\Imag{\mathrm{Im}}
\newcommand{\tran}{^{\text{\sf T}}}
\newcommand{\herm}{^{\text{\sf H}}}


% Solution environment
\definecolor{lightgray}{gray}{0.95}
\newmdenv[linecolor=white,backgroundcolor=lightgray,frametitle=Solution:]{solution}



\begin{document}

\title{Problems:  Coding and Capacity on Fading Channels\\
ECE-GY 6023. Wireless Communications}
\author{Prof.\ Sundeep Rangan}
\date{}

\maketitle


\begin{enumerate}

\item \emph{Slow vs.\ fast fading:}.
For each scenario below state whether the variations would likely be 
slow or fast fading relative to the coding block. 
Use reasonable assumptions and explain your reasoning.  There is no single correct answer.
\begin{enumerate}[label=(\alph*)]
\item A 5G NR base stations transmits over a channel with a \SI{100}{ns} delay spread,
to a UE moving at $v=$ \SI{30}{m/s} with a $180^\circ$ angular spread.
The carrier frequency is $f_c=$ \SI{28}{GHz}.
The transmission is over a \SI{100}{MHz} bandwidth in \SI{125}{\micro\second} slots.
\item A UAV is connected to a ground base station via point-to-point link with a line-of-sight.
So, there is no multipath fading.  But the UAV rotates 360$^\circ$ about once a second.
The beamwidth of the UAV 
antenna element is 60$^\circ$ and packets are transmitted once every \SI{1}{ms}.
\end{enumerate}

\item \emph{Error rate on uncoded modulation}:
\begin{enumerate}[label=(\alph*)]
\item Use any reference to find the symbol error rate (SER) of 16-QAM as
a function of the SNR $\gamma_s = E_s/N_0$.  Your expression will have a $Q$-function.
\item Find the SNR $\gamma_s$ requred for a SER of $(10)^{-3}$ assuming a constant channel.  
You can use MATLAB to invert the $Q$-function.
\item Suppose that the channel is Rayleigh fading, so $\gamma_s$ is exponentially distributed.
Find the average SNR, $\Exp(\gamma_s)$ so that the average SER is $(10)^{-3}$.
\end{enumerate}

\item \emph{Slow fading and outage probability:}  An access point is installed in an office
area with four rooms.  The path loss from the access point to each room and the percentage of
users in each room are as follows:
\begin{center}
\begin{tabular}{|c|c|c|}
  \hline
  % after \\: \hline or \cline{col1-col2} \cline{col3-col4} ...
  Room & Path loss [dB] & Fraction users \\ \hline
  1 & 60 & 0.6 \\ \hline
  2 & 80 & 0.3 \\ \hline
  3 & 90 & 0.06 \\ \hline
  4 & 100 & 0.04 \\ \hline
\end{tabular}
\end{center}
The AP has a transmit power of \SI{15}{dBm} and bandwidth of \SI{18}{MHz}.
The thermal noise at the receivers, including noise figure is \SI{-165}{dBm/Hz}.
\begin{enumerate}[label=(\alph*)]
\item If there is no fading, what SNR can be guaranteed to at least 95\% of the users?
\item Now suppose that, at each location, there is Rayleigh fading that can be modeled as flat
over the transmissions.
Write an expression for the CDF of the SNR including variation in both location and fading.
\item What is the SNR that can be guaranteed to at least 95\% of the users if we need
to account for slow fading?  You can use MATLAB to invert the expression in part (b).
\end{enumerate}

\item \emph{Ergodic capacity:}  A channel has two paths.  One path would be received
at power, $P_1$ and delay $\tau_1$, and the second path would be received at power
$P_2$ and delay $\tau_2$ where $\tau_2 > \tau_1$.  
Suppose you signal over a bandwidth $W \gg 1/(\tau_2 - \tau_1)$ and noise power spectral density
is $N_0$.
\begin{enumerate}[label=(\alph*)]
\item What is the average SNR over the band?
\item What is the ergodic capacity over the band?
\item Evaluate the expressions in (a) and (b) with  $P_1/(W N_0)=$ \SI{8}{dB}
and $P_2/(W N_0)=$ \SI{5}{dB}.
\end{enumerate}


\item \emph{LLRs:}  For each of the following channels, find the log likelihood ratio (LLR):
\[
    L(r) = \log \frac{ p(r|c=1) }{ p(r|c=0) }
\]
for the following channels:
\begin{enumerate}[label=(\alph*)]
\item Real-valued binary channel with fading:
\[
    r= Ax + w, \quad w \sim {\mathcal N}(0,N_0/2), \quad
        x = \begin{cases}
         \sqrt{E_x/2} & \mbox{if } c = 1, \\
        -\sqrt{E_x/2} & \mbox{if } c = 0.
    \end{cases}
\]
The LLR $L$ should depend on $A$ and $N_0$.

\item Binary symmetric channel:
\[
    r = c + w ~(\mbox{mod } 2), \quad w =
        \begin{cases}
        1 & \mbox{with probability } p \\
        0 & \mbox{with probability } 1-p
        \end{cases}
\]
Thus, $r \in \{0,1\}$ where there is a bit error with probability $p$.

\item Non-coherent channel:
\[
    r = \begin{cases}
        h + n & \mbox{when } c = 1\\
        n  & \mbox{when } c = 0,
        \end{cases}
        \quad h \sim {\mathcal CN}(0,E_s), ~n\sim {\mathcal CN}(0,N_0).
\]

\end{enumerate}

\item \emph{Bitwise likelihood}:  Suppose that two bits $(c_0,c_1)$ are modulated to
a $4$-PAM constellation
(the real or imaginary component of a 16-QAM constellation):
\[
    r = x + n, \quad n \sim {\mathcal N}(0,N_0/2),
\]
where the transmitted symbol
\[
    x = \begin{cases}
        -3A & \mbox{if } (c_0,c_1) = (00) \\
         -A & \mbox{if } (c_0,c_1) = (01) \\
          A & \mbox{if } (c_0,c_1) = (11) \\
         3A & \mbox{if } (c_0,c_1) = (10)
         \end{cases}
\]
Assume all the transmitted bits are equally likely.
\begin{enumerate}[label=(\alph*)]
\item Given a symbol energy, $E_s$, find $A$ such that $\Exp|x|^2 = E_s/2$.
\item Find the bitwise LLR for $c_0$:
\[
    L_0(r) = \log \frac{p(r|c_0=1)}{p(r|c_0=1)}.
\]
Use total probability
\[
    p(r|c_0) = \frac{1}{2}\left[ p(r|c_0,c_1=1) + p(r|c_0,c_1=0) \right].
\]
Find the bitwise LLR for $c_1$ as well.
\end{enumerate}

\item \emph{Row-column interleavers:}  One simple way of doing interleaving is as follows.
The input is a sequence of bits of length $MN$ for some parameters $M$ and $N$.
We read the bits into an $M \times N$ array, one row at a time.  Then, we read out
the bits one column at a time.  If two bits are adjacent on the input what is the minimum separation on the output?
\end{enumerate}

\end{document}

